
\begin{abstract}
Since its successful launch in June 2008, the {\it Fermi} Gamma-ray Space Telescope has made important breakthroughs in the understanding of the Gamma-Ray Burst (GRB) phenomemon.
The combination of the GBM and the LAT instruments onboard the {\it Fermi} observatory has provided a wealth of information from its observations of GRBs over seven decades in energy.
We present brief descriptions of the {\it Fermi} instruments and their capabilities for GRB science,
and report highlights from {\it  Fermi} observations of high-energy prompt and extended GRB emission.
The main physical implications of these results are discussed, along with open questions regarding GRB modelling.
We emphasize future synergies with ground-based \v Cerenkov telescopes at the time of the SVOM mission.

%{\it To cite this article: F. Piron, V. Connaughton, C. R. Physique x (20xx).}

\vskip 0.5\baselineskip
\selectlanguage{francais}
%\noindent{\bf R\'esum\'e}
\vskip 0.5\baselineskip
\noindent
{\bf Observations des sursauts gamma avec Fermi (r\'esum\'e)}
\vskip 0.5\baselineskip
Depuis sa mise en orbite en juin 2008, le t\'elescope spatial {\it Fermi} a permis des avanc\'ees remarquables dans la compr\'ehension des sursauts gamma.
La moisson de r\'esultats obtenus par {\it Fermi} a \'et\'e rendue possible par la combinaison des instruments \`a bord de l'observatoire, le GBM et le LAT, couvrant un domaine spectral s'\'etendant sur plus de sept ordres de grandeur en \'energie.
Nous r\'esumons les caract\'eristiques des deux instruments et leurs capacit\'es pour la d\'etection et l'\'etude des sursauts gamma, passons en revue les r\'esultats les plus marquants, et pr\'esentons leurs implications physiques imm\'ediates.
Apr\`es un rapide examen des questions soulev\'ees par ces observations et des enjeux th\'eoriques futurs, nous discutons les synergies observationnelles
avec les t\'elescopes qui seront op\'erationnels aux tr\`es hautes \'energies \`a l'\`ere de la mission SVOM.
%
% French title, abstract and keywords will be included later...
%{\it Pour citer cet article~: A. Name1, A. Name2, C. R.
%Physique 6 (2005).}
%Now keywords/mots-clÈs

%\keyword{Gamma-Ray Bursts; gamma rays; high energy; {\it Fermi}; GBM; LAT; prompt emission; extended emission; gamma-gamma opacity; bulk Lorentz factor; Extragalactic Background Light;
%quantum gravity; Lorentz invariance; particle acceleration; cascades; synchrotron radiation; inverse Compton radiation; ultra high-energy cosmic rays; very high energy; \v Cerenkov telescopes} \vskip %0.5\baselineskip
%\noindent{\small{\it Mots-cl\'es~:} sursa	uts gamma; rayons gamma; haute \'energie; {\it Fermi}; GBM; LAT; \'emission prompte; \'emission r\'emanente; opacit\'e gamma-gamma; facteur de Lorentz
%d'ensemble; fond diffus cosmique; gravit\'e quantique; invariance de Lorentz; acc\'el\'eration de particules; cascades; \'emission synchrotron; \'emission Compton inverse; rayons cosmiques d'ultra-
%haute \'energie; tr\`es haute \'energie; t\'elescopes \v Cerenkov
\keyword{Gamma-Ray Bursts; {\it Fermi}; bulk Lorentz factor; Extragalactic Background Light; Lorentz invariance; \v Cerenkov telescopes} \vskip 0.5\baselineskip
\noindent{\small{\it Mots-cl\'es~:} sursauts gamma; {\it Fermi}; facteur de Lorentz d'ensemble; fond diffus cosmique; invariance de Lorentz; t\'elescopes \v Cerenkov
}}
\end{abstract}
\end{frontmatter}